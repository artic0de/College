% ----------------------------------------------------------------
% AMS-LaTeX Paper ************************************************
% **** -----------------------------------------------------------
%\documentclass{amsart}
%\usepackage{txfonts}
%\documentclass[12pt,oneside]{article}
\documentclass{amsart}
\usepackage{graphicx}
\usepackage{enumitem}
% ----------------------------------------------------------------
\vfuzz2pt % Don't report over-full v-boxes if over-edge is small
\hfuzz2pt % Don't report over-full h-boxes if over-edge is small
% THEOREMS -------------------------------------------------------
\newtheorem{thm}{Theorem}[section]
\newtheorem{cor}[thm]{Corollary}
\newtheorem{lem}[thm]{Lemma}
\newtheorem{prop}[thm]{Proposition}
\theoremstyle{definition}
\newtheorem{defn}[thm]{Definition}
\theoremstyle{Exercise}
\newtheorem{ex}[thm]{Exercise}
\theoremstyle{remark}
\newtheorem{rem}[thm]{Remark}
\theoremstyle{rule}
\newtheorem{rul}[thm]{Rule}

\numberwithin{equation}{section}
% MATH -----------------------------------------------------------
\newcommand{\norm}[1]{\left\Vert#1\right\Vert}
\newcommand{\abs}[1]{\left\vert#1\right\vert}
\newcommand{\set}[1]{\left\{#1\right\}}
\newcommand{\Real}{\mathbb R}
\newcommand{\Z}{\mathbb Z}
\newcommand{\To}{\longrightarrow}
\newcommand{\BX}{\bB(X)}
\newcommand{\A}{\mathcal{A}}
% ----------------------------------------------------------------

% define some simple, commonly-used commands
\newcommand{\eps}{\varepsilon}
\newcommand{\dsum}{\displaystyle\sum}
\newcommand{\dint}{\displaystyle\int}

\newcommand{\pdr}[2]{\dfrac{\partial{#1}}{\partial{#2}}}
\newcommand{\pdrr}[2]{\dfrac{\partial^2{#1}}{\partial{#2}^2}}
\newcommand{\pdrt}[3]{\dfrac{\partial^2{#1}}{\partial{#2}{\partial{#3}}}}
\newcommand{\dr}[2]{\dfrac{d{#1}}{d{#2}}}
\newcommand{\aver}[1]{\langle {#1} \rangle}
\newcommand{\Baver}[1]{\Big\langle {#1} \Big\rangle}

\newcommand{\bzero}{\mathbf 0}
\newcommand{\bGamma}{\mbox{\boldmath{$\Gamma$}}}
\newcommand{\btheta}{\boldsymbol \theta}
\newcommand{\bchi}{\mbox{\boldmath{$\chi$}}}
\newcommand{\bnu}{\boldsymbol \nu}
\newcommand{\bmu}{\boldsymbol \mu}
\newcommand{\brho}{\mbox{\boldmath{$\rho$}}}
\newcommand{\bxi}{\boldsymbol \xi}
\newcommand{\bnabla}{\boldsymbol \nabla}
\newcommand{\bOm}{\boldsymbol \Omega}
\newcommand{\blambda}{\boldsymbol \lambda}
\newcommand{\bsigma}{\boldsymbol \sigma}

\newcommand{\bbR}{\mathbb R}
\newcommand{\bbC}{\mathbb C}
\newcommand{\bbQ}{\mathbb Q}
\newcommand{\bbN}{\mathbb N}
\newcommand{\bbZ}{\mathbb Z}

\newcommand{\ba}{\mathbf a} \newcommand{\bb}{\mathbf b}
\newcommand{\bc}{\mathbf c} \newcommand{\bd}{\mathbf d}
\newcommand{\be}{\mathbf e} \newcommand{\bff}{\mathbf f}
\newcommand{\bg}{\mathbf g} \newcommand{\bh}{\mathbf h}
\newcommand{\bi}{\mathbf i} \newcommand{\bj}{\mathbf j}
\newcommand{\bk}{\mathbf k} \newcommand{\bl}{\mathbf l}
\newcommand{\bm}{\mathbf m} \newcommand{\bn}{\mathbf n}
\newcommand{\bo}{\mathbf o} \newcommand{\bp}{\mathbf p}
\newcommand{\bq}{\mathbf q} \newcommand{\br}{\mathbf r}
\newcommand{\bs}{\mathbf s} \newcommand{\bt}{\mathbf t}
\newcommand{\bu}{\mathbf u} \newcommand{\bv}{\mathbf v}
\newcommand{\bw}{\mathbf w} \newcommand{\bx}{\mathbf x}
\newcommand{\by}{\mathbf y} \newcommand{\bz}{\mathbf z}
\newcommand{\bA}{\mathbf A} \newcommand{\bB}{\mathbf B}
\newcommand{\bC}{\mathbf C} \newcommand{\bD}{\mathbf D}
\newcommand{\bE}{\mathbf E} \newcommand{\bF}{\mathbf F}
\newcommand{\bG}{\mathbf G} \newcommand{\bH}{\mathbf H}
\newcommand{\bI}{\mathbf I} \newcommand{\bJ}{\mathbf J}
\newcommand{\bK}{\mathbf K} \newcommand{\bL}{\mathbf L}
\newcommand{\bM}{\mathbf M} \newcommand{\bN}{\mathbf N}
\newcommand{\bO}{\mathbf O} \newcommand{\bP}{\mathbf P}
\newcommand{\bQ}{\mathbf Q} \newcommand{\bR}{\mathbf R}
\newcommand{\bS}{\mathbf S} \newcommand{\bT}{\mathbf T}
\newcommand{\bU}{\mathbf U} \newcommand{\bV}{\mathbf V}
\newcommand{\bW}{\mathbf W} \newcommand{\bX}{\mathbf X}
\newcommand{\bY}{\mathbf Y} \newcommand{\bZ}{\mathbf Z}

\newcommand{\cA}{\mathcal A} \newcommand{\cB}{\mathcal B}
\newcommand{\cC}{\mathcal C} \newcommand{\cD}{\mathcal D}
\newcommand{\cE}{\mathcal E} \newcommand{\cF}{\mathcal F}
\newcommand{\cG}{\mathcal G} \newcommand{\cH}{\mathcal H}
\newcommand{\cI}{\mathcal I} \newcommand{\cJ}{\mathcal J}
\newcommand{\cK}{\mathcal K} \newcommand{\cL}{\mathcal L}
\newcommand{\cM}{\mathcal M} \newcommand{\cN}{\mathcal N}
\newcommand{\cO}{\mathcal O} \newcommand{\cP}{\mathcal P}
\newcommand{\cQ}{\mathcal Q} \newcommand{\cR}{\mathcal R}
\newcommand{\cS}{\mathcal S} \newcommand{\cT}{\mathcal T}
\newcommand{\cU}{\mathcal U} \newcommand{\cV}{\mathcal V}
\newcommand{\cW}{\mathcal W} \newcommand{\cX}{\mathcal X}
\newcommand{\cY}{\mathcal Y} \newcommand{\cZ}{\mathcal Z}


%%%%%%%%%%%%%%Start%%%%%%%%%%%%%Start%%%%%%%%%%%Start%%%%%%%%%%%%%%%Start%%%%%%%%%%%%%%%%%%%%%%%%%Start%%%%%%%%%%%%%%%%
%%%%%%%%%%%%%%Start%%%%%%%%%%%%%Start%%%%%%%%%%%Start%%%%%%%%%%%%%%%Start%%%%%%%%%%%%%%%%%%%%%%%%%Start%%%%%%%%%%%%%%%%
%%%%%%%%%%%%%%Start%%%%%%%%%%%%%Start%%%%%%%%%%%Start%%%%%%%%%%%%%%%Start%%%%%%%%%%%%%%%%%%%%%%%%%Start%%%%%%%%%%%%%%%%
%\documentclass[12pt,oneside]{article}

\usepackage{pdfpages}
%--------------
\usepackage{enumitem}
%-------------Tasks
%\usepackage{tasks} %\begin{tasks} \item \end{tasks}
%\bfseries Horizontal list: a = alphabetical \normalfont
%\begin{tasks}[counter-format = {tsk[a].},label-offset = {0.6em},label-format = {\bfseries}](6)
%\task One
%\task Two
%\task Three
%\task Four
%\task Five
%\task Six
%\task Seven
%\task Eight
%\task Nine
%\task Ten
%\end{tasks}
%\vglue5mm
%\bfseries Horizontal list: A = Alphabetical \normalfont
%\begin{tasks}[counter-format = {(tsk[A])},label-offset = {0.8em},label-format = {\bfseries}](3)
%\task One
%\task Two
%\task Three
%\task Four
%\task Five
%\task Six
%\task Seven
%\task Eight
%\task Nine
%\task Ten
%\end{tasks}



%___________________________
\usepackage[margin=2.5cm]{geometry}

\geometry{hmargin=3cm,vmargin=2cm}
\usepackage{tikz}
\def\width{18}
\def\hauteur{13}


\pagestyle{plain}

%%%%%%%%%%%%%%Start%%%%%%%%%%%%%Start%%%%%%%%%%%Start%%%%%%%%%%%%%%%Start%%%%%%%%%%%%%%%%%%%%%%%%%Start%%%%%%%%%%%%%%%%
%%%%%%%%%%%%%%Start%%%%%%%%%%%%%Start%%%%%%%%%%%Start%%%%%%%%%%%%%%%Start%%%%%%%%%%%%%%%%%%%%%%%%%Start%%%%%%%%%%%%%%%%
%%%%%%%%%%%%%%Start%%%%%%%%%%%%%Start%%%%%%%%%%%Start%%%%%%%%%%%%%%%Start%%%%%%%%%%%%%%%%%%%%%%%%%Start%%%%%%%%%%%%%%%%

\usepackage{fancyhdr}

\pagestyle{fancy}
\fancyhf{}
\rhead{}
\chead{\includegraphics[scale=.1]{snhu_logo.png}}
\begin{document}

\title{\sf MAT 230 Exam Two}%



%\thm{bbjh}


\begin{center}
\includegraphics[scale=.1]{snhu_logo.png}
\end{center}

%\thm{bbjh}
\maketitle
This document is proprietary to Southern New Hampshire University. It and the problems within may not be posted on any non-SNHU website.\\\\\\\\
\begin{center}
%Enter your name below this line:
Eric Trahan
\end{center}

\begin{center}
\rule{\textwidth}{0.4pt}
\end{center}
\newpage
\section*{}
\section*{}
Directions: Type your solutions into this document and be sure to show all steps for arriving at your solution. Just giving a final number may not receive full credit.
\\
\section*{Problem 1}

A website reports that 70\% of its users are from outside a certain country, and 60\% of its users log on the website every day. Suppose that for its users from inside the country, that 80\% of them log on every day. What is the probability that a person is from the country given that he logs on the website every day? Has the probability that he is from the country increased or decreased with the additional information?
\\\\
  %Enter your answer below this comment line.  
  If 70\% of its users are from outside the country, then 30\% are from inside the country.\\
  So the probability that a user is from inside the country is 3/10.\\
  If 80\% of the users from inside the country log on every day, then that is 80\% of 30\%, which is $.8 * 30\% = 24\%$\\
  That is to say, there is a 24\% chance, or a 6 out of 25 chance, that a user is both from inside the country AND logs on every day.\\
  This does decrease the probability that the user is from the country, since we are filtering even further.
\\\\
\newpage
  \section*{Problem 2}
 This question has 2 parts.
 The drawing below shows a Hasse diagram for a partial order on the set:
 \\\\
   $\{A, \;B,\; C,\; D,\; E,\; F,\; G,\; H,\; I, \; J\}$
 \includegraphics[width=2.5in]{hass}
\\\\
 {\color{blue} {\bf Figure 1:} \emph{A Hasse diagram shows 10 vertices and 8 edges. The vertices, represented by dots, are as follows: vertex J; vertices H and I are aligned vertically to the right of vertex J; vertices A, B, C, D, and E form a closed loop, which is to the right of vertices H and I; vertex G is inclined upward to the right of vertex E; and vertex F is inclined downward to the right of vertex E. The edges, represented by line segments, between the vertices are as follows: vertex J is connected to no vertex; a vertical edge connects vertices H and I; a vertical edge connects vertices B and C; and 6 inclined edges connect the following vertices, A and B, C and D, D and E, A and E, E and G, and E and F.
  }
  }
  \\
 \subsection*{Part 1:}
 Determine the properties of the Hasse diagram based on the following questions:

  \begin{enumerate}[label=(\alph*)]
    \item What are the minimal elements of the partial order?
\\\\
  %Enter your answer below this comment line.  
  J, I, A, F
\\\\
    \item What are the maximal elements of the partial order?
\\\\
  %Enter your answer below this comment line.  
  J, H, D, G
\\\\
    \item Which of the following pairs are comparable?
\[(A,\, D),\; (J,\, F),\; (B,\, E),\; (G,\, F),\; (D,\, B),\; (C,\, F),\; (H,\, I), (C,\, E)\]
\\\\
  %Enter your answer below this comment line.  
  \[(A,\, D),\; (G,\, F),\; (D,\, B),\; (H,\, I)\]
\\\\
   \end{enumerate}
   \newpage

\subsection*{Part 2:}
Consider the partial order with domain $\{3,\, 5,\, 6, \,7,\, 10,\, 14,\, 20,\, 30,\,60\}$ and with $x\,\leq \,y$ if $x$ evenly divides $y$. Select the correct Hasse diagram for the partial order.

\begin{enumerate}[label=(\roman*)]
\item
\fbox{
\includegraphics[width=5in]{hass3}\\

}
\\\\
{\color{blue}{\bf Figure 2:} \emph{A Hasse diagram shows a set of elements, 3, 5, 6, 7, 10, 14, 20, 30, and 60. The elements are arranged as follows: 3, 6, and 30 are arranged vertically one above another, connected by vertical line segments; 5 and 20 are arranged vertically one above another to the right of 3 and 6, respectively, connected by a vertical line segment; 10 is placed inclined up to the right of 20, horizontally aligned with 30, and 60 is placed above 30 and 10, vertically aligned with 20; at the top; inclined line segments connect 10 and 30, 30 and 60, 60 and 20, 20 and 5, 5 and 10, and 5 and 30 to form a diamond shape; and 7 and 14 are placed to the right of 5 and 10, respectively, connected by a vertical line segment.
}
}
\\
\\
  %Enter your answer below this comment line.  
\\\\
\newpage
\item
\fbox{
 \includegraphics[width=5in]{hass2}
}
\\\\
{\color{blue}{\bf Figure 3:} \emph{A Hasse diagram shows a set of elements, 3, 5, 6, 7, 10, 14, 20, 30, and 60. The elements are arranged as follows: 3, 6, and 30 are arranged vertically one above another on the left, connected by vertical line segments; 5, 10, and 20 are arranged vertically one above another on the right, connected by vertical line segments; 60 is placed at the top, connected to 30 and 20 by inclined line segments; and 7 and 14 are placed to the right of 10 and 20, respectively, connected by a vertical line segment.
}
}
\\
\\
  %Enter your answer below this comment line.  
\\\\
\newpage
\item
\fbox{
 \includegraphics[width=5in]{hass1}\\
}
\\\\
{\color{blue}{\bf Figure 4:} \emph{A Hasse diagram shows a set of elements, 3, 5, 6, 7, 10, 14, 20, 30, and 60. The elements are arranged as follows: 3 at the bottom; 6 is placed vertically above 3; 5 is placed to the right and little above 3; 10 is placed vertically above 5; 7 is placed to the right of 5 but little above 10; 30, 20, 10, and 60 are placed in a diamond together, and 60 is placed at the top. The vertical line segments between the elements are as follows: 3 and 6, 6 and 30, 5 and 10, and 7 and 14. The inclined line segments between the elements are as follows: 10 and 30, 10 and 20, 30 and 60, and 20 and 60.
}
}
\\
\\
  %Enter your answer below this comment line.  
  This is the Hasse diagram that correctly shows the partial order with domain $\{3,\, 5,\, 6, \,7,\, 10,\, 14,\, 20,\, 30,\,60\}$ and with $x\,\leq \,y$ if $x$ evenly divides $y$.
\\\\
\newpage
\item
\fbox{
\includegraphics[width=5in]{hass4}
}
\\\\
{\color{blue}{\bf Figure 5:} \emph{A Hasse diagram shows a set of elements, 3, 5, 6, 7, 10, 14, 20, 30, and 60. The elements are arranged as follows: 5, 10, 20, 30, and 60 are arranged vertically one above another, connected by vertical line segments; 3 and 6 are arranged vertically one above another, to the right of 10 and 20, respectively, connected by a vertical line segment; an inclined line segment connects 30 and 6; and 7 and 14 are placed to the left of 10 and 20, respectively, connected by a vertical line segment.
}
}
\\\\
  %Enter your answer below this comment line.  
\\\\

\end{enumerate}
  \newpage
  \section*{Problem 3}
  You just bought a new car from a dealership that sells cars no older than 2015. Consider the relationship to be the ``same age.'' The relation is ``same age'' and the set is cars made in 2015, 2016, 2017, 2018, 2019, and 2020.

  \begin{enumerate}[label=(\alph*)]
    \item Prove that this relation is an equivalence relation.
\\\\
  %Enter your answer below this comment line.  
  In order for a relationship to be equivalence, it must meet the following requirements:\\
  It must be reflexive\\
  It must be symmetric\\
  It must be transitive\\
  \\
  This relation is reflexive, because all cars can be the 'same age' as themselves.\\
  This relation is symmetric, because if car $a$ is the same age as car $b$, then the opposite MUST be true.\\
  This relation is transitive, because if car $a$ is the same age as car $b$, and car $b$ is the same age as car $c$, then car $a$ MUST be the same age as car $c$.\\
  \\
  Therefore, this is an equivalence relation.
  \\\\
    \item Describe the partition defined by the equivalence classes.
\\\\
  %Enter your answer below this comment line.  
  Since the relation is same age, each year is it's own separate equivalence class.\\
  Because the domain is the set of cars made in 2015, 2016, 2017, 2018, 2019, and 2020, the partition, whose union equals the domain, is also the set of cars made in 2015, 2016, 2017, 2018, 2019, and 2020.\\
  This is mainly because there is no relation between each year of cars.
\\\\
  \end{enumerate}
\newpage
  \section*{Problem 4}
 Analyze each graph and explain why the graph does or does not have an Euler circuit. If it does, specify the nodes within the circuit.
  \begin{enumerate}[label=(\roman*)]
\item 
\fbox{
 \includegraphics[width=4in]{euler1}
}
\\\\
{\color{blue} {\bf Figure 6:} \emph{An undirected graph has 6 vertices, a through f. 5 vertices are in the form of a regular pentagon, rotated 90 degrees clockwise. Hence, the top vertex becomes the rightmost vertex. From the bottom left vertex, moving clockwise, the vertices in the pentagon shape are labeled: a, b, c, e, and f. Vertex d is above vertex e, below and to the right of vertex c. Undirected edges, line segments, are between the following vertices: b and c; b and a; b and f; b and e; a and c; a and d; a and f; c and d; c and f; d and e; and d and f.
  }
}\\\\
  %Enter your answer below this comment line.  
  Euler circuit (in example order of connectivity):\\
  (e, d, c, b, a, f, d, a, c, f, b, e)\\
  It is a euler circuit because each edge is used only once, it starts and ends at the same vertex, and every vertex and edge is used.
\\\\
   \newpage
\item
\fbox{
\includegraphics[width=4in]{euler2}
}
\\\\
{\color{blue} {\bf Figure 7:} \emph{
An undirected graph has 6 vertices, a through f. 5 vertices are in the form of a regular pentagon, rotated 90 degrees clockwise. Hence, the top vertex becomes the rightmost vertex. From the bottom left vertex, moving clockwise, the vertices in the pentagon shape are labeled: a, b, c, e, and f. Vertex d is above vertex e, below and to the right of vertex c. Undirected edges, line segments, are between the following vertices: a and b; a and c; a and d; a and f; c and d; d and f; d and e; b and e; and b and f.
  }
}
\\\\
  %Enter your answer below this comment line.  
  This graph does not represent an euler circuit, since euler circuits cannot have vertices with an odd number degree.\\
  Both vertices f and b have a degree of 3.
\\\\
   \newpage
\item 
\fbox{
\includegraphics[width=4in]{euler3}
}
\\\\
{\color{blue} {\bf Figure 8: } \emph{A graph with five vertices, a, b, c, d,  and e.\\ Vertex a is connected to no vertex.\\
  Vertex b is connected to c and e.\\
  Vertex c is connected to b and d.\\
  Vertex d is connected to c and e.\\
  Vertex e is connected to b and d.\\
  }
}
\\\\
  %Enter your answer below this comment line.  
  This is not an euler circuit, since every vertex must be used.\\
  There is no vertex connected to vertex a, therefore a cannot be used in a circuit.
\\\\

\newpage
\fbox{
 \includegraphics[width=4in]{euler4}
}
\\\\
\item {\color{blue} {\bf Figure 9:} \emph{An undirected graph has 7 vertices, a through g. 5 vertices are in the form of a regular pentagon, rotated 90 degrees clockwise. Hence, the top vertex becomes the rightmost vertex. From the bottom left vertex, moving clockwise, the vertices in the pentagon shape are labeled: a, b, c, e, and f. Vertex d is above vertex e, below and to the right of vertex c. Vertex g is below vertex e, above and to the right of vertex f. Undirected edges, line segments, are between the following vertices: a and b; a and c; a and f; b and f; b and c; c and d; c and g; d and e; d and f; and f and g.
  }
}
\\\\
  %Enter your answer below this comment line.  
  This graph does not represent an euler circuit since it has several vertexes with odd degrees.\\
  e, d, b, and a are all vertexes with odd degrees.
\\\\


  \end{enumerate}
\newpage
  \section*{Problem 5}
  Use Prim’s algorithm to compute the minimum spanning tree for the weighted graph. Start the algorithm at vertex A. Show the order in which the edges are added to the tree.
\\
\includegraphics[width=5in]{prim}
\\\\
{\color{blue} {\bf Figure 10:} \emph{A weighted graph shows 5 vertices, represented by circles, and 6 edges, represented by line segments. Vertices A, B, C, and D are placed at the corners of a rectangle, whereas vertex E is at the center of the rectangle. The edges, A B, B D, A C, C D, A E, and E C, have the weights, 7, 3, 2, 4, 5, and 6, respectively.
  }
}
\\\\
  %Enter your answer below this comment line.  
  First we start at A.\\
  Since (A,C) has a smaller weight than (A,B), we add (A,C) to our spanning tree.\\
  Next is (C,D).\\
  Now (B,D).\\
  Lastly, (A,E).\\
  Each path was chosen because it was the smallest weight option.\\
  The total weight is: $2 + 4 + 3 + 5 = 14$
\\\\
 \newpage
  \section*{Problem 6}
A lake initially contains 1000 fish. Suppose that in the absence of predators or other causes of removal, the fish population increases by 10\% each month. However, factoring in all causes, 80 fish are lost each month.\\

Give a recurrence relation for the population of fish after $n$ months. How many fish are there after 5 months? If your fish model predicts a non-integer number of fish, round down to the next lower integer.
\\\\
  %Enter your answer below this comment line.  
  First we define the base case, $f_0$ which is 1000.\\
  Next, we define the sequence, which is $f_n = f_{n-1} * 1.10 - 80$, since each amount is increased by ten percent and loses 80 fish.\\
  \\
  After 5 months, the fish population would be as follows:\\
  $f_0 = 1000$\\
  $f_1 = 1000 * 1.10 - 80 = 1020$\\
  $f_2 = 1020 * 1.10 - 80 = 1042$\\
  $f_3 = 1042 * 1.10 - 80 = 1066.2$\\
  $f_4 = 1066.2 * 1.10 - 80 = 1092.82$\\
  $f_5 = 1092.82 * 1.10 - 80 = 1122.102 \approx 1122$ fish total after 5 months.
\\\\
\end{document}

