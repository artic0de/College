% ----------------------------------------------------------------
% AMS-LaTeX Paper ************************************************
% **** -----------------------------------------------------------
\documentclass{amsart}
%\usepackage{txfonts}
\usepackage{graphicx}
\usepackage{enumitem}
\usepackage{amsmath}
\usepackage{amssymb}
% ----------------------------------------------------------------
\vfuzz2pt % Don't report over-full v-boxes if over-edge is small
\hfuzz2pt % Don't report over-full h-boxes if over-edge is small
% THEOREMS -------------------------------------------------------
\newtheorem{thm}{Theorem}[section]
\newtheorem{cor}[thm]{Corollary}
\newtheorem{lem}[thm]{Lemma}
\newtheorem{prop}[thm]{Proposition}
\theoremstyle{definition}
\newtheorem{defn}[thm]{Definition}
\theoremstyle{Exercise}
\newtheorem{ex}[thm]{Exercise}
\theoremstyle{remark}
\newtheorem{rem}[thm]{Remark}
\theoremstyle{rule}
\newtheorem{rul}[thm]{Rule}

\numberwithin{equation}{section}
% MATH -----------------------------------------------------------
\newcommand{\norm}[1]{\left\Vert#1\right\Vert}
\newcommand{\abs}[1]{\left\vert#1\right\vert}
\newcommand{\set}[1]{\left\{#1\right\}}
\newcommand{\Real}{\mathbb R}
\newcommand{\Z}{\mathbb Z}
\newcommand{\To}{\longrightarrow}
\newcommand{\BX}{\bB(X)}
\newcommand{\A}{\mathcal{A}}
% ----------------------------------------------------------------

% define some simple, commonly-used commands
\newcommand{\eps}{\varepsilon}
\newcommand{\dsum}{\displaystyle\sum}
\newcommand{\dint}{\displaystyle\int}

\newcommand{\pdr}[2]{\dfrac{\partial{#1}}{\partial{#2}}}
\newcommand{\pdrr}[2]{\dfrac{\partial^2{#1}}{\partial{#2}^2}}
\newcommand{\pdrt}[3]{\dfrac{\partial^2{#1}}{\partial{#2}{\partial{#3}}}}
\newcommand{\dr}[2]{\dfrac{d{#1}}{d{#2}}}
\newcommand{\aver}[1]{\langle {#1} \rangle}
\newcommand{\Baver}[1]{\Big\langle {#1} \Big\rangle}

\newcommand{\bzero}{\mathbf 0}
\newcommand{\bGamma}{\mbox{\boldmath{$\Gamma$}}}
\newcommand{\btheta}{\boldsymbol \theta}
\newcommand{\bchi}{\mbox{\boldmath{$\chi$}}}
\newcommand{\bnu}{\boldsymbol \nu}
\newcommand{\bmu}{\boldsymbol \mu}
\newcommand{\brho}{\mbox{\boldmath{$\rho$}}}
\newcommand{\bxi}{\boldsymbol \xi}
\newcommand{\bnabla}{\boldsymbol \nabla}
\newcommand{\bOm}{\boldsymbol \Omega}
\newcommand{\blambda}{\boldsymbol \lambda}
\newcommand{\bsigma}{\boldsymbol \sigma}

\newcommand{\bbR}{\mathbb R}
\newcommand{\bbC}{\mathbb C}
\newcommand{\bbQ}{\mathbb Q}
\newcommand{\bbN}{\mathbb N}
\newcommand{\bbZ}{\mathbb Z}

\newcommand{\ba}{\mathbf a} \newcommand{\bb}{\mathbf b}
\newcommand{\bc}{\mathbf c} \newcommand{\bd}{\mathbf d}
\newcommand{\be}{\mathbf e} \newcommand{\bff}{\mathbf f}
\newcommand{\bg}{\mathbf g} \newcommand{\bh}{\mathbf h}
\newcommand{\bi}{\mathbf i} \newcommand{\bj}{\mathbf j}
\newcommand{\bk}{\mathbf k} \newcommand{\bl}{\mathbf l}
\newcommand{\bm}{\mathbf m} \newcommand{\bn}{\mathbf n}
\newcommand{\bo}{\mathbf o} \newcommand{\bp}{\mathbf p}
\newcommand{\bq}{\mathbf q} \newcommand{\br}{\mathbf r}
\newcommand{\bs}{\mathbf s} \newcommand{\bt}{\mathbf t}
\newcommand{\bu}{\mathbf u} \newcommand{\bv}{\mathbf v}
\newcommand{\bw}{\mathbf w} \newcommand{\bx}{\mathbf x}
\newcommand{\by}{\mathbf y} \newcommand{\bz}{\mathbf z}
\newcommand{\bA}{\mathbf A} \newcommand{\bB}{\mathbf B}
\newcommand{\bC}{\mathbf C} \newcommand{\bD}{\mathbf D}
\newcommand{\bE}{\mathbf E} \newcommand{\bF}{\mathbf F}
\newcommand{\bG}{\mathbf G} \newcommand{\bH}{\mathbf H}
\newcommand{\bI}{\mathbf I} \newcommand{\bJ}{\mathbf J}
\newcommand{\bK}{\mathbf K} \newcommand{\bL}{\mathbf L}
\newcommand{\bM}{\mathbf M} \newcommand{\bN}{\mathbf N}
\newcommand{\bO}{\mathbf O} \newcommand{\bP}{\mathbf P}
\newcommand{\bQ}{\mathbf Q} \newcommand{\bR}{\mathbf R}
\newcommand{\bS}{\mathbf S} \newcommand{\bT}{\mathbf T}
\newcommand{\bU}{\mathbf U} \newcommand{\bV}{\mathbf V}
\newcommand{\bW}{\mathbf W} \newcommand{\bX}{\mathbf X}
\newcommand{\bY}{\mathbf Y} \newcommand{\bZ}{\mathbf Z}

\newcommand{\cA}{\mathcal A} \newcommand{\cB}{\mathcal B}
\newcommand{\cC}{\mathcal C} \newcommand{\cD}{\mathcal D}
\newcommand{\cE}{\mathcal E} \newcommand{\cF}{\mathcal F}
\newcommand{\cG}{\mathcal G} \newcommand{\cH}{\mathcal H}
\newcommand{\cI}{\mathcal I} \newcommand{\cJ}{\mathcal J}
\newcommand{\cK}{\mathcal K} \newcommand{\cL}{\mathcal L}
\newcommand{\cM}{\mathcal M} \newcommand{\cN}{\mathcal N}
\newcommand{\cO}{\mathcal O} \newcommand{\cP}{\mathcal P}
\newcommand{\cQ}{\mathcal Q} \newcommand{\cR}{\mathcal R}
\newcommand{\cS}{\mathcal S} \newcommand{\cT}{\mathcal T}
\newcommand{\cU}{\mathcal U} \newcommand{\cV}{\mathcal V}
\newcommand{\cW}{\mathcal W} \newcommand{\cX}{\mathcal X}
\newcommand{\cY}{\mathcal Y} \newcommand{\cZ}{\mathcal Z}

%%%%%%%%%%%%%%Start%%%%%%%%%%%%%Start%%%%%%%%%%%Start%%%%%%%%%%%%%%%Start%%%%%%%%%%%%%%%%%%%%%%%%%Start%%%%%%%%%%%%%%%%
%%%%%%%%%%%%%%Start%%%%%%%%%%%%%Start%%%%%%%%%%%Start%%%%%%%%%%%%%%%Start%%%%%%%%%%%%%%%%%%%%%%%%%Start%%%%%%%%%%%%%%%%
%%%%%%%%%%%%%%Start%%%%%%%%%%%%%Start%%%%%%%%%%%Start%%%%%%%%%%%%%%%Start%%%%%%%%%%%%%%%%%%%%%%%%%Start%%%%%%%%%%%%%%%%
\usepackage{fancyhdr}

\pagestyle{fancy}
\fancyhf{}
\rhead{}
\chead{\includegraphics[scale=.1]{snhu_logo.png}}


\begin{document}
\title{\sf Module 2 Problem Set}

\begin{center}
\includegraphics[scale=.1]{snhu_logo.png}
\end{center}

%\thm{bbjh}
\maketitle
This document is proprietary to Southern New Hampshire University. It and the problems within may not be posted on any non-SNHU website.\\\\\\\\
\begin{center}
%Enter your name below this line:
Eric Trahan
\end{center}


\begin{center}
\rule{\textwidth}{0.4pt}
\end{center}


\newpage


\newpage

\section*{}
\section*{}

Directions: Type your solutions into this document and be sure to show all steps for arriving at your solution. Just giving a final number may not receive full credit.\\

\section*{Problem 1}
\subsection*{Part 1}
{\bf Indicate whether the argument is valid or invalid. For valid arguments, prove that the argument is valid using a truth table. For invalid arguments, give truth values for the variables showing that the argument is not valid.}\\
 \begin{enumerate}

\item \[
\begin{array}{||c||}
\hline \hline
(p \land q) \to r\\
\\
\therefore (p \lor q) \to r\\
\hline \hline
\end{array}
\]\\\\
 %Enter your answer below this comment line.
 The argument is invalid.\\
 \begin{displaymath}
 \begin{array}{c|c c c|c|c|}
 # & p & q & r & (p \land q) \to r & (p \lor q) \to r\\
 \hline
 1. & T & T & T & T & T\\
 2. & T & T & F & F & F\\
 3. & T & F & T & T & T\\
 4. & T & F & F & T & F\\
 5. & F & T & T & T & T\\
 6. & F & T & F & T & F\\
 7. & F & F & T & T & T\\
 8. & F & F & F & T & T\\
 \end{array}
 \end{displaymath}
 For validity, we are looking for the truth value of the conclusion when all hypotheses are true. If any conclusions are false when the hypotheses are true, then the statement is invalid. There are two rows where the argument is proven to be invalid, rows 4 and 6. In both cases, the hypothesis is true, but the conclusion is false, making the argument invalid.
 
 \\\\
   \end{enumerate}

\subsection*{Part 2}
{\bf Converse and inverse errors are typical forms of invalid arguments. Prove that each argument is invalid by giving truth values for the variables showing that the argument is invalid. You may find it easier to find the truth values by constructing a truth table.}\\
 \begin{enumerate}[label=(\alph*)]
\item Converse error
\[
\begin{array}{||c||}
\hline \hline
p \to q\\
q\\
\\
\therefore p\\
\hline \hline
\end{array}
\]\\\\
%Enter your answer below this comment line.
\begin{displaymath}
 \begin{array}{c|c c|c|}
 # & p & q & p \to q\\
 \hline
 1. & T & T & T\\
 2. & T & F & F\\
 3. & F & T & T\\
 4. & F & F & T\\
 \end{array}
 \end{displaymath}
 Again, we're looking for the truth value of the conclusion where all hypotheses are true. All hypotheses are true in rows 1 and 3. However, the conclusion is false on row 3, making this argument invalid.
 \\\\
\item Inverse error
\[
\begin{array}{||c||}
\hline \hline
p \to q\\
\neg p\\
\\
\therefore \neg q\\
\hline \hline
\end{array}
\]\\\\
%Enter your answer below this comment line.
\begin{displaymath}
 \begin{array}{c|c c|c|}
 # & p & q & p \to q\\
 \hline
 1. & T & T & T\\
 2. & T & F & F\\
 3. & F & T & T\\
 4. & F & F & T\\
 \end{array}
 \end{displaymath}
 The first hypothesis is true on lines 1, 3, and 4. The second hypothesis is true on lines 3 and 4, so we will only be looking for validity in lines 3 and 4 since on those rows both hypotheses are true. The argument is proved invalid on line 3, since both hypotheses are true and the conclusion, $\neg q$, resolves to false.
\\\\
\end{enumerate}

\subsection*{Part 3}
{\bf Which of the following arguments are invalid and which are valid? Prove your answer by replacing each proposition with a variable to obtain the form of the argument. Then prove that the form is valid or invalid.}\\
 \begin{enumerate}[label=(\alph*)]
  \item \[
\begin{array}{||c||}
\hline \hline
\text{The patient has high blood pressure or diabetes or both.}\\
\text {The patient has diabetes or high cholesterol or both.}\\
\\
\therefore \text {The patient has high blood pressure or high cholesterol.
}\\
\hline \hline
\end{array}
\]\\\\\
%Enter your answer below this comment line.
Let $p = $the patient.\\
$B(x) = x$ has high blood pressure.\\
$D(x) = x$ has diabetes.\\
$C(x) = x$ has high cholesterol.\\\\
The argument can now be shown as:\\
$B(p) \lor D(p)$\\
$D(p) \lor C(p)$\\
--------------------\\
$\therefore B(p) \lor C(p)$\\\\
Now lets look at a truth table:
\begin{displaymath}
 \begin{array}{c|c c c|c|c|c|}
 # & B(p) & D(p) & C(p) & B(p) \lor D(p) & D(p) \lor C(p) & B(p) \lor C(p)\\
 \hline
 1. & T & T & T & T & T & T\\
 2. & T & T & F & T & T & T\\
 3. & T & F & T & T & T & T\\
 4. & T & F & F & T & F & T\\
 5. & F & T & T & T & T & T\\
 6. & F & T & F & T & T & F\\
 7. & F & F & T & F & T & T\\
 8. & F & F & F & F & F & F\\
 \end{array}
 \end{displaymath}
 \\
 We're only concerned with where all hypotheses are true, so lines 1, 2, 3, 5, and 6. In all of these cases, except line 6, the argument would appear to be valid. On line 6 however, both hypotheses are true, but the conclusion is false, thus rendering the argument invalid.
\\\\\

 
    \end{enumerate}
 \newpage
%--------------------------------------------------------------------------------------------------

\section*{Problem 2}
\subsection*{Part 1}

 Which of the following arguments are valid? Explain your reasoning.\\
 \begin{enumerate}[label=(\alph*)]
\item I have a student in my class who is getting an $A$. Therefore, John, a student in my class, is getting an $A$. \\\\
%Enter your answer below this comment line.
Invalid. All we know is that some student is getting an A. There is no way to prove that John specifically is getting an A. Even if we try, we get:\\
1. $\exists x A(x) $where $A(x)$ = $x$ is getting an A  - Hypothesis\\
2. $A(s)$ for some student $s$.   - Existential instantiation, 1\\
3. $\exists x A(x)$ - Existential generation, 2\\
Because we can only implement existential generation on $A(s)$ since it uses a particular element.
\\\\
\item Every Girl Scout who sells at least 30 boxes of cookies will get a prize. Suzy, a Girl Scout, got a prize. Therefore, Suzy sold at least 30 boxes of cookies.\\\\
%Enter your answer below this comment line.
Invalid. This is because a girl scout can still receive a prize even if she doesn't sell at least 30 boxes of cookies, according to conditional logic. This means that if C(x):' x sold at least 30 boxes of cookies' is false, but P(x):' x got a prize' is true, then both hypotheses are true while the condition is false, making the argument invalid.
\\\\
 \end{enumerate}

 \subsection*{Part 2}
Determine whether each argument is valid. If the argument is valid, give a proof using the laws of logic. If the argument is invalid, give values for the predicates $P$ and $Q$ over the domain ${a,\; b}$ that demonstrate the argument is invalid.\\
 \begin{enumerate}[label=(\alph*)]
\item \[
\begin{array}{||c||}
\hline \hline
\exists x\, (p(x)\; \land \;Q(x) )\\
\\
\therefore \exists x\, Q(x)\; \land\; \exists P(x) \\
\hline \hline
\end{array}
\]\\\\
 %Enter your answer here.
 It is valid:\\
 1. $\exists x (P(x) \land Q(x))$   Hypothesis\\
 2. $P(c) \land Q(c)$   Existential Instantiation, 1\\
 3. $Q(c) \land P(c)$   Commutative, 2\\
 4. $Q(c)$  Simplification, 3\\
 5. $P(c)$  Simplification, 3\\
 6. $\exists x Q(x)$    Existential Generalization, 4\\
 7. $\exists x P(x)$    Existential Generalization, 5\\
 8. $\exists x Q(x) \land \exists x P(x)$   Conjunction, 4, 5
 \\\\


\item \[
\begin{array}{||c||}
\hline \hline
\forall x\, (p(x)\; \lor \;Q(x) )\\
\\
\therefore \forall x\, Q(x)\; \lor \; \forall P(x) \\
\hline \hline
\end{array}
\]\\\\
 %Enter your answer here.
 It is invalid. This is because we know at least one of the predicates is true, but the other could be false, so we cannot simplify the disjunction. This means we cannot separate $\forall x (P(x) \lor Q(x))$ to reform it as $\forall x Q(x) \lor \forall x P(x)$.\\\\
 Example:\\
 1. $\forall x (P(x) \lor Q(x))$    Hypothesis\\
     2. $P(a) \lor Q(a)$    Universal Instantiation.\\
 We cannot continue, as we would need to separate/simplify the disjunction, and since either $P(a)$ or $Q(a)$ could be false, we cannot determine which predicate to simplify to... much less use both predicates.
 \\\\
 \end{enumerate}
 \newpage
%--------------------------------------------------------------------------------------------------


\section*{Problem 3}

Prove the following using a direct proof. Your proof should be expressed in complete English sentences.
\\\\

If $a$, $b$, and $c$ are integers such that $b$ is a multiple of $a^3$ and $c$ is a multiple of $b^2$, then $c$ is a multiple of $a^6$.
\\\\
%Enter your answer below this comment line.
\begin{proof}
Assume $a$, $b$, and $c$ are integers such that $b$ is a multiple of $a^3$ and $c$ is a multiple of $b^2$, we will prove $c$ is a multiple of $a^6$.\\
If $b$ is a multiple of $a^3$, then $a^3 = b*k$ for some integer $k$.\\
Next, we solve for $b$ by dividing both sides by $k$, yielding $b = a^3/k$.\\
If $c$ is a multiple of $b^2$, then $b^2 = c*j$ for some integer $j$.\\
We can plug in $a^3/k$ for $b$, giving $(a^3/k)^2 = c*j$.\\
Next, we solve $(a^3/k)^2$, which gives us $a^6/k^2 = c*j$\\
Finally, we multiply both sides by $k^2$, which gives us $a^6 = c*j*k^2$, proving that $c$ is a multiple of $a^6$.
\end{proof}


\\\\

 \newpage
%--------------------------------------------------------------------------------------------------
\section*{Problem 4}
Prove the following using a direct proof:
\\

The sum of the squares of 4 consecutive integers is an even integer.\\\\
%Enter your answer below this comment line.
\begin{proof}
Let $a$, $b$, $c$, and $d$ be any four consecutive integers.
We will prove that the sum of the squares of these four consecutive integers is an even integer. In other words, $a^2 + b^2 + c^2 + d^2 = g$, where $g$ is an even integer.\\
If these four integers are consecutive, then two of them must be even, and two of them must be odd.\\
Assume $a$ and $c$ are even, $b$ and $d$ are odd.\\
If $a$ and $c$ are even, then they are the product of two and any integer.\\
Therefore, $a = 2w$ for any integer $w$, and $c = 2y$ for any integer $y$.\\
If $b$ and $d$ are odd, then they are the sum of one plus the product of two and any integer.\\
Therefore, $b = 2x + 1$ for any integer $x$, and $d = 2y + 1$ for any integer $y$.\\
Now we plug in our new values, and we get $(2w)^2 + (2x + 1)^2 + (2y)^2 + (2z + 1)^2 = g$.\\
The square of any even integer is even, and the square of any odd integer is odd. We can show this by simplifying the problem to the following:\\
$(4w^2) + (4(x^2 + x) + 1) + (4y^2) + (4(z^2 + z) + 1) = g$.\\
The sum of two odd integers is even, and the sum of two even integers is even. We will show this by putting the sums in the form of the product between two and any number.\\
For the two odd integers, we combine them and put them in the form $2*integer$:\\
$(4(x^2 + x) + 1) + (4(z^2 + z) + 1) = 4(x^2 + x) + 4(z^2 + z) + 2 = 2(2(x^2 + x) + 2(z^2 + z) + 1)$.\\
For the two even integers, we combine them and put them in the form $2*integer$:\\
$(4w^2) + (4y^2) = 2(2w^2 + 2y^2)$.\\
Now we are left with $2(2w^2 + 2y^2) + 2(2(x^2 + x) + 2(z^2 + z) + 1) = g$, which is the sum of two even integers, which we already proved to be even. Therefore, we will not prove it again, but we will show the final equation:\\ $2((2w^2 + 2y^2) + (2(x^2 + x) + 2(z^2 + z) + 1)) = g$.\\
Since $g$ is in the form $2 * integer$: $g$, which is the sum of the squares of 4 consecutive integers, is proven to be even.
\end{proof}
\\\\


 \newpage
%--------------------------------------------------------------------------------------------------
\section*{Problem 5}

Prove the following using a proof by contrapositive:
\\\\

For every pair of real numbers $x$ and $y$, if $x$ is rational and $xy$ is irrational, then $y$ is irrational.\\
%Enter your answer below this comment line.
\begin{proof}
Let $x$ and $y$ be a pair of real numbers, if $y$ is rational, then $xy$ is rational or $x$ is irrational.\\
Since $y$ is rational, we can show it as $\frac{a}{b}$ for any integers $a$ and $b$ where $b \not= 0$.\\
In the case of proving $xy$ is rational, $x$ has to be rational and expressible by $\frac{c}{d}$ for any integers $c$ and $d$ where $d \not= 0$.\\
This would mean that $\frac{a}{b} * \frac{c}{d} = xy$.\\
If we simplify, we get $\frac{ac}{bd} = xy$, and since the product of any two integers is an integer, this means that $xy$ is equal to any integer over any integer where the denominator is not 0, making it rational.\\
In the case of proving $x$ is irrational, $xy$ has to be irrational. We will represent the irrational product of $xy$ with $p$.\\
If $y$ is rational, it is still expressible by $\frac{a}{b}$, therefore $x * \frac{a}{b} = p$.\\
Solving for $x$, we get $x = \frac{bp}{a}$. Since $p = xy$ and $xy$ must be irrational, $\frac{bp}{a}$ must also be irrational.\\
Therefore, $x$ is irrational.\\
\end{proof}


\\\\




 \newpage
%--------------------------------------------------------------------------------------------------

\section*{Problem 6}
Prove the following using a proof by contradiction:
\\\\


The average of four real numbers is greater than or equal to at least one of the numbers.\\
%Enter your answer below this comment line.
\begin{proof}
Suppose $a$, $b$, $c$, and $d$ are all real numbers. We will attempt to prove that the average of the four numbers is less than any of the numbers.\\
That is to say, $(a + b + c + d)/4 < e$ where $e$ is the smallest of the four numbers.\\
First, we will multiply both sides by four, yielding $a + b + c + d < 4e$.\\
Next, we will subtract the smallest number from both sides. We will assume it's $d$ for our purposes, but our selection of variables here isn't what is important, it's that the smallest number is being subtracted from $4e$ (four times the smallest of the four numbers), which will yield $3e$.\\
So now we have $a + b + c < 3e$.\\
Next, we will divide both sides by 3, giving us $(a + b + c)/3 < e$, the left of which is the average of the three biggest numbers and the right side is the smallest of the first four numbers.\\
This is illogical, and proves our hypothesis false. The average of the biggest three numbers cannot be less than the smallest of the first four numbers. Even if they were all the same numbers, it would be $(1 + 1 + 1)/3 < 1$, which is $1 < 1$, and that is false.\\ 
\end{proof}
\\\\



 \newpage
%--------------------------------------------------------------------------------------------------

\section*{Problem 7}

Let $\displaystyle q = \frac{a}{b}$ and $\displaystyle r = \frac{c}{d}$ be two rational numbers written in lowest terms. Let $s = q + r$ and $\displaystyle s = \frac{e}{f}$ be written in lowest terms. Assume that $s$ is not $0$.\\

 Prove or disprove the following two statements.
\\\\
a.  If $b$ and $d$ are odd, then $f$ is odd.
\\\\
b. If $b$ and $d$ are even, then $f$ is even.
\\\\
%Enter your answer below this comment line.
Answer a.\\
If $b$ and $d$ are odd, then we end up with $s = \frac{a}{2k + 1} + \frac{c}{2j + 1}$ for any integers $j$ and $k$.\\
Simplifying the equation, we get $s = \frac{a(2j + 1) + c(2k + 1)}{(2k+1)(2j+1)}$.\\
Simplifying even further, we get $s = \frac{a(2j + 1) + c(2k + 1)}{2k(2j+1)+2j+1}$.\\
Simplifying again, $s = \frac{a(2j + 1) + c(2k + 1)}{4kj+2k+2j+1}$.\\
And once more, but this time to find even and odd numbers: $s = \frac{a(2j + 1) + c(2k + 1)}{2((2kj)+k)+(2j+1)}$.\\
So now we have, in the denominator, the sum between an even number and an odd number.\\
Now let's prove that $f$ is odd with $f = 2((2kj)+k)+(2j+1)$.\\
First we subtract one from both sides and then factor out two, which yields the following: $f - 1 = 2(((2kj)+k)+j)$.\\
Next we add one to both sides and get: $f = 2(((2kj)+k)+j) + 1$.\\
Since $f$ is equal to the form $2*integer + 1$, this proves that $f$ is odd when $b$ and $d$ are odd.
\\\\
Answer b.\\
If $b$ and $d$ are even, then they can be represented as two times any integer, thus $\frac{e}{f} = \frac{a}{2k} + \frac{c}{2j}$ for any integers $k$ and $j$.\\
Like before, we simplify until we can determine even and odd numbers in the denominator:\\
$\frac{e}{f} = \frac{2aj + 2ck}{2j*2k} = \frac{e}{f} = \frac{2aj + 2ck}{2(jk)}$.\\
Ignoring the numerators, we have $f = 2(jk)$, which is two times any integer, proving that $f$ is even when $b$ and $d$ are even.
\\\\


\newpage

\section*{Problem 8}
{\bf Define $P(n)$ to be the assertion that:}\\
\[\displaystyle \sum_{j=1}^{n}\, j^2 \;=\;\frac{n(n+1)(2n+1)}{6}\]\\\\
\begin{enumerate}[label=(\alph*)]
  \item Verify that $P(3)$ is true.\\\\
   %Enter your answer here.
   First, for the left side: $1^2 + 2^2 + 3^2 = 14$.\\
   Then the right side: $\frac{3(3+1)(2(3)+1)}{6} = \frac{12(7)}{6} = \frac{84}{6} = 14$.\\
   Therefore, $P(3)$ is true since both the left and right sides equal 14.\\
  \item Express $P(k)$.\\\\
   %Enter your answer here.
   $P(k): \sum_{j=1}^{k} j^2 = \frac{k(k+1)(2k+1)}{6}$.\\
  \item Express $P(k+1)$.\\\\
   %Enter your answer here.
   $P(k+1): \sum_{j=1}^{k+1} j^2 = \frac{(k+1)(k+2)(2k+3)}{6}$.\\
   \item In an inductive proof that for every positive integer $n$,
   \[\displaystyle \sum_{j=1}^{n}\, j^2 \;=\;\frac{n(n+1)(2n+1)}{6}\]
   what must be proven in the base case?\\\\
    %Enter your answer here.
    The base case must prove $P(1): \sum_{j=1}^{1} j^2 = \frac{1((1)+1)(2(1)+1)}{6}$.\\
   \item What would be the inductive hypothesis in the inductive step from your previous answer?\\\\
    %Enter your answer here.
    For any integers $k \geq 1$, if $\sum_{j=1}^{k} j^2 = \frac{k(k+1)(2k+1)}{6}$, then $\sum_{j=1}^{k+1} j^2 = \frac{(k+1)(k+2)(2k+3)}{6}$.\\
   \item Prove by induction that for any positive integer n,
   \[\displaystyle \sum_{j=1}^{n}\, j^2 \;=\;\frac{n(n+1)(2n+1)}{6}\]
    %Enter your answer here.
    \\
    \begin{proof}
    By induction on $n$.\\\\
    Base case: $n=1$.\\\\
    When $n=1$, the left side of the equation is $\sum_{j=1}^{1} j^2 = 1$.\\\\
    When $n=1$, the right side of the equation is $\frac{1((1)+1)(2(1)+1)}{6} = \frac{(2)(3)}{6} = 1$.\\\\
    Therefore, $\sum_{j=1}^{1} j^2 = \frac{1((1)+1)(2(1)+1)}{6}$.\\
    \\
    Inductive step: Suppose that for positive integer $k$, $\sum_{j=1}^{k} j^2 = \frac{k(k+1)(2k+1)}{6}$, then we will show that $\sum_{j=1}^{k+1} j^2 = \frac{(k+1)(k+2)(2k+3)}{6}$.\\\\
    We can interpret the formula like so: $\sum_{j=1}^{k+1} j^2 = \sum_{j=1}^{k} j^2 + (k + 1)$.\\\\
    From what we have proven already, we can present the equation like so: $\sum_{j=1}^{k+1} j^2 = \sum_{j=1}^{k} j^2 + (k + 1) = \frac{k(k+1)(2k+1)}{6} + (k + 1)$.\\\\
    Now we can simplify for: $\sum_{j=1}^{k+1} j^2 = \frac{k(k+1)(2k+1)}{6} + (k + 1) = \frac{(k+1)(k+2)(2k+3)}{6}$.\\\\
    This proves that the proposition will hold for all positive integer $n$.\\ 
    \end{proof}
\end{enumerate}

\end{document}
