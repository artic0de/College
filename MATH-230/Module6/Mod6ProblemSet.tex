% ----------------------------------------------------------------
% AMS-LaTeX Paper ************************************************
% **** -----------------------------------------------------------
%\documentclass{amsart}
%\usepackage{txfonts}
%\documentclass[12pt,oneside]{article}
\documentclass{amsart}
\usepackage{graphicx}
\usepackage{enumitem}
\usepackage{xcolor}
% ----------------------------------------------------------------
\vfuzz2pt % Don't report over-full v-boxes if over-edge is small
\hfuzz2pt % Don't report over-full h-boxes if over-edge is small
% THEOREMS -------------------------------------------------------
\newtheorem{thm}{Theorem}[section]
\newtheorem{cor}[thm]{Corollary}
\newtheorem{lem}[thm]{Lemma}
\newtheorem{prop}[thm]{Proposition}
\theoremstyle{definition}
\newtheorem{defn}[thm]{Definition}
\theoremstyle{Exercise}
\newtheorem{ex}[thm]{Exercise}
\theoremstyle{remark}
\newtheorem{rem}[thm]{Remark}
\theoremstyle{rule}
\newtheorem{rul}[thm]{Rule}

\numberwithin{equation}{section}
% MATH -----------------------------------------------------------
\newcommand{\norm}[1]{\left\Vert#1\right\Vert}
\newcommand{\abs}[1]{\left\vert#1\right\vert}
\newcommand{\set}[1]{\left\{#1\right\}}
\newcommand{\Real}{\mathbb R}
\newcommand{\Z}{\mathbb Z}
\newcommand{\To}{\longrightarrow}
\newcommand{\BX}{\bB(X)}
\newcommand{\A}{\mathcal{A}}
% ----------------------------------------------------------------

% define some simple, commonly-used commands
\newcommand{\eps}{\varepsilon}
\newcommand{\dsum}{\displaystyle\sum}
\newcommand{\dint}{\displaystyle\int}

\newcommand{\pdr}[2]{\dfrac{\partial{#1}}{\partial{#2}}}
\newcommand{\pdrr}[2]{\dfrac{\partial^2{#1}}{\partial{#2}^2}}
\newcommand{\pdrt}[3]{\dfrac{\partial^2{#1}}{\partial{#2}{\partial{#3}}}}
\newcommand{\dr}[2]{\dfrac{d{#1}}{d{#2}}}
\newcommand{\aver}[1]{\langle {#1} \rangle}
\newcommand{\Baver}[1]{\Big\langle {#1} \Big\rangle}

\newcommand{\bzero}{\mathbf 0}
\newcommand{\bGamma}{\mbox{\boldmath{$\Gamma$}}}
\newcommand{\btheta}{\boldsymbol \theta}
\newcommand{\bchi}{\mbox{\boldmath{$\chi$}}}
\newcommand{\bnu}{\boldsymbol \nu}
\newcommand{\bmu}{\boldsymbol \mu}
\newcommand{\brho}{\mbox{\boldmath{$\rho$}}}
\newcommand{\bxi}{\boldsymbol \xi}
\newcommand{\bnabla}{\boldsymbol \nabla}
\newcommand{\bOm}{\boldsymbol \Omega}
\newcommand{\blambda}{\boldsymbol \lambda}
\newcommand{\bsigma}{\boldsymbol \sigma}

\newcommand{\bbR}{\mathbb R}
\newcommand{\bbC}{\mathbb C}
\newcommand{\bbQ}{\mathbb Q}
\newcommand{\bbN}{\mathbb N}
\newcommand{\bbZ}{\mathbb Z}

\newcommand{\ba}{\mathbf a} \newcommand{\bb}{\mathbf b}
\newcommand{\bc}{\mathbf c} \newcommand{\bd}{\mathbf d}
\newcommand{\be}{\mathbf e} \newcommand{\bff}{\mathbf f}
\newcommand{\bg}{\mathbf g} \newcommand{\bh}{\mathbf h}
\newcommand{\bi}{\mathbf i} \newcommand{\bj}{\mathbf j}
\newcommand{\bk}{\mathbf k} \newcommand{\bl}{\mathbf l}
\newcommand{\bm}{\mathbf m} \newcommand{\bn}{\mathbf n}
\newcommand{\bo}{\mathbf o} \newcommand{\bp}{\mathbf p}
\newcommand{\bq}{\mathbf q} \newcommand{\br}{\mathbf r}
\newcommand{\bs}{\mathbf s} \newcommand{\bt}{\mathbf t}
\newcommand{\bu}{\mathbf u} \newcommand{\bv}{\mathbf v}
\newcommand{\bw}{\mathbf w} \newcommand{\bx}{\mathbf x}
\newcommand{\by}{\mathbf y} \newcommand{\bz}{\mathbf z}
\newcommand{\bA}{\mathbf A} \newcommand{\bB}{\mathbf B}
\newcommand{\bC}{\mathbf C} \newcommand{\bD}{\mathbf D}
\newcommand{\bE}{\mathbf E} \newcommand{\bF}{\mathbf F}
\newcommand{\bG}{\mathbf G} \newcommand{\bH}{\mathbf H}
\newcommand{\bI}{\mathbf I} \newcommand{\bJ}{\mathbf J}
\newcommand{\bK}{\mathbf K} \newcommand{\bL}{\mathbf L}
\newcommand{\bM}{\mathbf M} \newcommand{\bN}{\mathbf N}
\newcommand{\bO}{\mathbf O} \newcommand{\bP}{\mathbf P}
\newcommand{\bQ}{\mathbf Q} \newcommand{\bR}{\mathbf R}
\newcommand{\bS}{\mathbf S} \newcommand{\bT}{\mathbf T}
\newcommand{\bU}{\mathbf U} \newcommand{\bV}{\mathbf V}
\newcommand{\bW}{\mathbf W} \newcommand{\bX}{\mathbf X}
\newcommand{\bY}{\mathbf Y} \newcommand{\bZ}{\mathbf Z}

\newcommand{\cA}{\mathcal A} \newcommand{\cB}{\mathcal B}
\newcommand{\cC}{\mathcal C} \newcommand{\cD}{\mathcal D}
\newcommand{\cE}{\mathcal E} \newcommand{\cF}{\mathcal F}
\newcommand{\cG}{\mathcal G} \newcommand{\cH}{\mathcal H}
\newcommand{\cI}{\mathcal I} \newcommand{\cJ}{\mathcal J}
\newcommand{\cK}{\mathcal K} \newcommand{\cL}{\mathcal L}
\newcommand{\cM}{\mathcal M} \newcommand{\cN}{\mathcal N}
\newcommand{\cO}{\mathcal O} \newcommand{\cP}{\mathcal P}
\newcommand{\cQ}{\mathcal Q} \newcommand{\cR}{\mathcal R}
\newcommand{\cS}{\mathcal S} \newcommand{\cT}{\mathcal T}
\newcommand{\cU}{\mathcal U} \newcommand{\cV}{\mathcal V}
\newcommand{\cW}{\mathcal W} \newcommand{\cX}{\mathcal X}
\newcommand{\cY}{\mathcal Y} \newcommand{\cZ}{\mathcal Z}


%%%%%%%%%%%%%%Start%%%%%%%%%%%%%Start%%%%%%%%%%%Start%%%%%%%%%%%%%%%Start%%%%%%%%%%%%%%%%%%%%%%%%%Start%%%%%%%%%%%%%%%%
%%%%%%%%%%%%%%Start%%%%%%%%%%%%%Start%%%%%%%%%%%Start%%%%%%%%%%%%%%%Start%%%%%%%%%%%%%%%%%%%%%%%%%Start%%%%%%%%%%%%%%%%
%%%%%%%%%%%%%%Start%%%%%%%%%%%%%Start%%%%%%%%%%%Start%%%%%%%%%%%%%%%Start%%%%%%%%%%%%%%%%%%%%%%%%%Start%%%%%%%%%%%%%%%%
\usepackage{fancyhdr}

\pagestyle{fancy}
\fancyhf{}
\rhead{}
\chead{\includegraphics[scale=.1]{snhu_logo.png}}

\begin{document}
\begin{center}
\includegraphics[scale=.1]{snhu_logo.png}
\end{center}
\title{\sf Module Six Problem Set}%



\maketitle
This document is proprietary to Southern New Hampshire University. It and the problems within may not be posted on any non-SNHU website.
\\\\\\\\
\begin{center}
%Enter your name below this line:
Eric Trahan
\end{center}


\begin{center}
\rule{\textwidth}{0.4pt}
\end{center}
\newpage


\section*{}
\section*{}
Directions: Type your solutions into this document and be sure to show all steps for arriving at your solution. Just giving a final number may not receive full credit.
\\

%--------------------------------------------------------------------------------------------------

\section*{Problem 1}


Indicate if each of the two graphs are equal.\\
 \begin{enumerate}[label=(\alph*)]
 

  \item
   \fbox{
\includegraphics[width=1.5in]{M6-fig1}

}
\hfil
  \fbox{

\includegraphics[width=1.5in]{M6-fig2}

}
\\\\
 {\color{blue}{\bf Figure 1:} \emph{Left: An undirected graph has 5 vertices. The vertices are arranged in the form of an inverted pentagon. From the top left vertex, moving clockwise, the vertices are labeled: a, b, c, d, and e. Undirected edges, line segments, are between the following vertices: a and b; a and c; b and c; c and d; e and d; and e and c. \\
 }
 }\\
{\color{blue}{\bf Figure 2:} \emph{
  Right: The adjacency list representation of a graph. The list shows all the vertices, a through e, in a column from top to bottom. The adjacent vertices for each vertex in the column are placed in a row to the right of the corresponding vertex’s cell in the column. An arrow points from each cell in the column to its corresponding row on the right. Data from the list, as follows: Vertex a is adjacent to vertices b and c. Vertex b is adjacent to vertices a and c. Vertex c is adjacent to vertices a, b, d, and e. Vertex d is adjacent to vertices c and e. Vertex e is adjacent to vertices c and d.
}
}
\\
\\
%Enter your answer below this comment line.
The graphs are both equal. If we made an adjacency list of the graph on the left, it would look exactly like the one on the right. If we made a graph with the adjacency list on the right, it could look exactly like the graph on the left.
\\\\
\newpage
 \item
  \fbox{
   \includegraphics[width=1.5in]{M6-fig3}
}
\hfil
  \fbox{
$
\left( \begin{array}{ccccc}
0 & 0 & 1 & 1 & 0 \\
0 & 0 & 0 & 0 & 1\\
1 & 0 & 0 & 1 & 0\\
1 & 0 & 1 & 0 & 1\\
0 & 1 & 0 & 1 & 0
\end{array} \right)
$
}\\\\
\\
\\
   {\color{blue}{\bf Figure 3:} \emph{An undirected graph has 5 vertices. The vertices are arranged in the form of an inverted pentagon. Moving clockwise from the top left vertex a, the other vertices are, b, c, d, and e. Undirected edges, line segments, are between the following vertices: a and c; a and d; d and c; and e and b. 
  \\
}
}
\\\\
%Enter your answer below this comment line.
They are not equal. The binary representation of a graph on the right implies that vertices $e$ and $d$ are connected, but they are not connected in the graph on the left.
\\\\


 \newpage

 \
%--------------------------------------------------------------------------------------------------

Prove that the two graphs below are isomorphic.\\\\

\item
   \fbox{
\includegraphics[width=2in]{M6-fig4}
}\\\\
 {\color{blue}{\bf Figure 4:} \emph{Two undirected graphs. Each graph has 6 vertices. The vertices in the first graph are arranged in two rows and 3 columns. From left to right, the vertices in the top row are 1, 2, and 3. From left to right, the vertices in the bottom row are 6, 5, and 4. Undirected edges, line segments, are between the following vertices: 1 and 2; 2 and 3; 1 and 5; 2 and 5; 5 and 3; 2 and 4; 3 and 6; 6 and 5; and 5 and 4. The vertices in the second graph are a through f. Vertices d, a, and c, are vertically inline. Vertices e, f, and b, are horizontally to the right of vertices d, a, and c, respectively. Undirected edges, line segments, are between the following vertices: a and d; a and c; a and e; a and b; d and b; a and f; e and f; c and f; and b and f.
}
}
\\
\\
%Enter your answer below this comment line.
Let's start with an adjacency chart for each graph:\\\\
Graph 1:\\
$1 \rightarrow 2,5$\\
$2 \rightarrow 1,3,4,5$\\
$3 \rightarrow 2,5,6$\\
$4 \rightarrow 2,5$\\
$5 \rightarrow 1,2,3,4,6$\\
$6 \rightarrow 3,5$\\
\\
Graph 2:\\
$a \rightarrow b,c,d,e,f$\\
$b \rightarrow a,d,f$\\
$c \rightarrow a,f$\\
$d \rightarrow a,b$\\
$e \rightarrow a,f$\\
$f \rightarrow a,b,c,e$\\
\\
If they are isomorphic, then we can assume there is some relationship between the two.\\
Let's assume the following and see if it holds true:\\
$5 \rightarrow a$\\
$2 \rightarrow f$\\
$3 \rightarrow b$\\
$4 \rightarrow e$\\
$1 \rightarrow c$\\
$6 \rightarrow d$\\
\\
Now we replace the letters in graph two with the identified numbers in our assumed relationship, and see if it's the same as graph one:\\
Graph 2:\\
$a \rightarrow b,c,d,e,f = 5 \rightarrow 1,2,3,4,6$\\
$b \rightarrow a,d,f = 3 \rightarrow 2,5,6$\\
$c \rightarrow a,f = 1 \rightarrow 2,5$\\
$d \rightarrow a,b = 6 \rightarrow 3,5$\\
$e \rightarrow a,f = 4 \rightarrow 2,5$\\
$f \rightarrow a,b,c,e = 2 \rightarrow 1,3,4,5$\\
\\
As we can see, our relationship proves to be true, and the graphs are in fact isomorphic.
\\\\

Show that the pair of graphs are not isomorphic by showing that there is a property that is preserved under isomorphism which one graph has and the other does not.\\

\item
\fbox{
\includegraphics[width=2in]{M6-fig5}
}\\\\
{\color{blue}{\bf Figure 5:} \emph{Two undirected graphs. The first graph has 5 vertices, in the form of a regular pentagon. From the top vertex, moving clockwise, the vertices are labeled: 1, 2, 3, 4, and 5. Undirected edges, line segments, are between the following vertices: 1 and 2; 2 and 3; 3 and 4; 4 and 5; and 5 and 1. The second graph has 4 vertices, a through d. Vertices d and c are horizontally inline, where vertex d is to the left of vertex c. Vertex a is above and between vertices d and c. vertex b is to the right and below vertex a, but above the other two vertices. Undirected edges, line segments, are between the following vertices: a and b; b and c; a and d; d and c; d and b.
}
}
\\
\\
%Enter your answer below this comment line.
The number of vertices and the number of edges are two properties that are preserved under isomorphism.\\
Graph one has 5 vertices and 5 edges, graph two has 4 vertices and 5 edges.\\
Since there is an unequal number of vertices, the two graphs are NOT isomorphic. 
\\\\

\end{enumerate}    
    
 \newpage

 
%--------------------------------------------------------------------------------------------------

\section*{Problem 2}    
    
Refer to the undirected graph provided below:
\\\\
  \fbox{

\includegraphics[width=2in]{M6-fig6}

}\\\\
{\color{blue}{\bf Figure 6:} \emph{An undirected graph has 9 vertices. 6 vertices form a hexagon, which is tilted upward to the right. Starting from the leftmost vertex, moving clockwise, the vertices forming the hexagon shape are: D, A, B, E, I, and F. Vertex H is above and to the right of vertex B. Vertex G is the rightmost vertex, below vertex H and above vertex E. Vertex C is the bottommost vertex, a little to the right of vertex E. Undirected edges, line segments, are between the following vertices: A and D; A and B; B and F; B and H; H and G; G and E; B and E; A and E; E and I; I and C; I and F; and F and D.
}
}
\\
\\
    \begin{enumerate}[label=(\roman*)]
        \item What is the maximum length of a path in the graph? Give an example of a path of that length.\\\\
           %Enter your answer here.
           8: (E, G), (G, H), (H, B), (B, A), (A, D), (D, F), (F, I), (I, C)
\\\\
        \item What is the maximum length of a cycle in the graph? Give an example of a cycle of that length.\\\\
           %Enter your answer here.
           8: (E, G), (G, H), (H, B), (B, A), (A, D), (D, F), (F, I), (I, E)
\\\\
        \item Give an example of an open walk of length five in the graph that is a trail but not a path.\\\\
           %Enter your answer here.
           1. (E, G)\\
           2. (G, H)\\
           3. (H, B)\\
           4. (B, E)\\
           5. (E, A)
\\\\
        \item Give an example of a closed walk of length four in the graph that is not a circuit.\\\\
           %Enter your answer here.
           1. (C, I)\\
           2. (I, E)\\
           3. (E, I)\\
           4. (I, C)
\\\\
        \item Give an example of a circuit of length zero in the graph.\\\\
           %Enter your answer here.
           (C, C)
\\\\
    \end{enumerate}
    \newpage
    

%--------------------------------------------------------------------------------------------------

\section*{Problem 3}

\begin{enumerate}[label=(\alph*)]
\item Find the connected components of each graph.\\
    \begin{enumerate}[label=(\roman*)]
    \item $G = (V, \,E).\quad V = \{a,\, b,\, c,\, d,\,  e\}.\quad E = \emptyset$\\\\
%Enter your answer below this comment line.
If there are no edges, then each vertex is a connected component. 
\\\\
    \item $G = (V,\, E).\quad V = \{a,\, b,\, c,\, d,\, e,\, f\}.\quad E = \{ \{c,\, f\}, \,\{a,\, b\},\, \{d,\, a\}, \,\{e,\, c\},\, \{b,\, f\} \}$\\\\
%Enter your answer below this comment line.
Connected component 1: \{a, b, c, d, e, f\}. It's the whole set, because there is an edge that connects every vertex to a singular component. 
\\\\
    \end{enumerate}
\item Determine the edge connectivity and the vertex connectivity of each graph.\\

    \begin{enumerate}[label=(\roman*)]    
 \item
\fbox{

\includegraphics[width=2in]{M6-fig7}

}
\\\\
{\color{blue}{\bf Figure 7:} \emph{An undirected graph has 8 vertices, 1 through 8. 4 vertices form a rectangular-shape on the left. Starting from the top left vertex and moving clockwise, the vertices of the rectangular shape are, 1, 2, 3, and 4. 3 vertices form a triangle on the right, with a vertical side on the left and the other vertex on the extreme right. Starting from the top vertex and moving clockwise, the vertices of the triangular shape are, 7, 8, and 5. Vertex 6 is between the rectangular shape and the triangular shape. Undirected edges, line segments, are between the following vertices: 1 and 2; 2 and 3; 3 and 4; 4 and 1; 2 and 6; 4 and 6; 3 and 6; 6 and 7; 6 and 8; 6 and 5; 7 and 5; 7 and 8; and 5 and 8.
\\
}
}
\\
\\
%Enter your answer below this comment line.
Edge connectivity: 2\\
Proof: Edges \{1, 2\} and \{1, 4\}\\\\
Vertex connectivity: 1\\
Proof: Vertex 6
\\\\
\newpage
\item  

\fbox{

\includegraphics[width=2in]{M6-fig8}

}
\\\\
{\color{blue}{\bf Figure 8:} \emph{An undirected graph has 8 vertices, 1 through 8. 4 vertices form a rectangular shape in the center. Starting from the top left vertex and moving clockwise, the vertices of the rectangular shape are, 3, 7, 5, and 6. Vertex 2 is at about the center of the rectangular shape. Vertex 8 is to the right of the rectangular shape. Vertex 1 and 4 are to the left of the rectangular shape, horizontally in-line with vertices 3 and 6, respectively. Undirected edges, line segments, are between the following vertices: 1 and 3; 3 and 7; 3 and 4; 3 and 6; 3 and 2; 4 and 2; 4 and 6; 6 and 2; 6 and 5; 2 and 5; 2 and 7; 2 and 8; 7 and 5; 7 and 8; and 5 and 8. 
\\
}}
\\
\\
%Enter your answer below this comment line.
Edge connectivity: 1\\
Proof: Edge \{1, 3\}\\\\
Vertex connectivity: 1\\
Proof: Vertex 3
\\\\
    \end{enumerate}
\end{enumerate}



 \newpage
 

\section*{Problem 4}
For each graph below, find an Euler circuit in the graph or explain why the graph does not have an Euler circuit.\\
\begin{enumerate}[label=(\alph*)]
\item
\fbox{

\includegraphics[width=2in]{M6-fig9}\\

}\\\\
{\color{blue}{\bf Figure 9:} \emph{An undirected graph has 6 vertices, a through f. 5 vertices are in the form of a regular pentagon, rotated 90 degrees clockwise. Hence, the top vertex becomes the rightmost vertex. From the bottom left vertex, moving clockwise, the vertices in the pentagon shape are labeled: a, b, c, e, and f. Vertex d is above vertex e, below and to the right of vertex c. Undirected edges, line segments, are between the following vertices: b and c; b and a; b and f; b and e; a and c; a and d; a and f; c and d; c and f; d and e; and d and f.
\\
}
}
\\
\\
%Enter your answer below this comment line.
Euler circuit path:\\
(e, d, c, b, a, f, d, a, c, f, b, e)
\\\\

\newpage
\item
\fbox{
 \includegraphics[width=2in]{M6-fig10}
}\\\\
{\color{blue}{\bf Figure 10:} \emph{An undirected graph has 7 vertices, a through g. 5 vertices are in the form of a regular pentagon, rotated 90 degrees clockwise. Hence, the top vertex becomes the rightmost vertex. From the bottom left vertex, moving clockwise, the vertices in the pentagon shape are labeled: a, b, c, e, and f. Vertex d is above vertex e, below and to the right of vertex c. Vertex g is below vertex e, above and to the right of vertex f. Undirected edges, line segments, are between the following vertices: a and b; a and c; a and d; a and f; b and f; b and c; b and e; c and d; c and g; d and e; d and f; and f and g.
\\
}
}
\\
\\
%Enter your answer below this comment line.
Euler circuit path:\\
(e, d, c, b, a, f, g, c, a, d, f, b, e)
\\\\

\newpage

\item
For each graph below, find an Euler trail in the graph or explain why the graph does not have an Euler trail.\\

{\it (Hint: One way to find an Euler trail is to add an edge between two vertices with odd degree, find an Euler circuit in the resulting graph, and then delete the added edge from the circuit.)}\\
\begin{enumerate}[label=(\roman*)]
\item
\fbox{

\includegraphics[width=2in]{M6-fig11}\\

}
\\\\
{\color{blue}{\bf Figure 11:} \emph{An undirected graph has 6 vertices, a through f. 5 vertices are in the form of a regular pentagon, rotated 90 degrees clockwise. Hence, the top vertex becomes the rightmost vertex. From the bottom left vertex, moving clockwise, the vertices in the pentagon shape are labeled: a, b, c, e, and f. Vertex d is above vertex e, below and to the right of vertex c. Undirected edges, line segments, are between the following vertices: a and b; a and c; a and d; a and f; b and f; b and c; c and d; c and f; d and e; and d and f.
}
}
\\\\
%Enter your answer below this comment line.
Euler trail:\\
(e, d, f, a, b, c, d, a, c, f, b)
\\\\
\item \fbox{

\includegraphics[width=2in]{M6-fig12}\\

}
\\\\
{\color{blue}{\bf Figure 12:} \emph{An undirected graph has 6 vertices, a through f. 5 vertices are in the form of a regular pentagon, rotated 90 degrees clockwise. Hence, the top vertex becomes the rightmost vertex. From the bottom left vertex, moving clockwise, the vertices in the pentagon shape are labeled: a, b, c, e, and f. Vertex d is above vertex e, below and to the right of vertex c. Undirected edges, line segments, are between the following vertices: a and b; a and c; a and d; a and f; b and f; b and c; b and e; c and d; d and e; and d and f. Edges c f, a d, and b e intersect at the same point.
\\
}
}
\\\\
%Enter your answer below this comment line.
Since every vertex has an even degree, there is no euler trail.
\end{enumerate}
\end{enumerate}
 \newpage
%--------------------------------------------------------------------------------------------------

\section*{Problem 5}

Consider the following tree for a prefix code:\\
\fbox{
\includegraphics[width=2in]{M6-fig13}\\
}
\\\\
{\color{blue}{\bf Figure 13:} \emph{A tree with 5 vertices. The top vertex branches into character, a, on the left, and a vertex on the right. The vertex in the second level branches into character, e, on the left, and a vertex on the right. The vertex in the third level branches into two vertices. The left vertex in the fourth level branches into character, c, on the left, and character, n, on the right. The right vertex in the fourth level branches into character, d, on the left, and character, y, on the right. The weight of each edge branching left from a vertex is 0. The weight of each edge branching right from a vertex is 1.
\\
}
}
\\
\\

\begin{enumerate}[label=(\alph*)]
\item Use the tree to encode ``day''.\\\\
%Enter your answer below this comment line.
111001111
\\\\
\item Use the tree to encode ``candy''.\\\\
%Enter your answer below this comment line.
11000110111101111
\\\\
\item Use the tree to decode $``1110101101''$.\\\\
%Enter your answer below this comment line.
den
\\\\
\item Use the tree to decode $``111001101110010''$.\\\\
%Enter your answer below this comment line.
dance
\\\\

\end{enumerate}

 \newpage
%--------------------------------------------------------------------------------------------------

\section*{Problem 6}

\fbox{
\includegraphics[width=2in]{M6-fig14}\\
}\\\\
{\color{blue}{\bf Figure 14:} \emph{A tree diagram has 9 vertices. The top vertex is d. Vertex d has three branches to vertices, f, b, and a. Vertex b branches to three vertices, i, h, and e. Vertex a branches to vertex c. Vertex c branches to vertex g.
\\
}
}
\\
\\
\begin{enumerate}[label=(\alph*)]
\item Give the order in which the vertices of the tree are visited in a post-order traversal.\\\\
%Enter your answer below this comment line.
(f, i, h, e, b, g, c, a, d)
\\\\
\item Give the order in which the vertices of the tree are visited in a pre-order traversal.\\\\
%Enter your answer below this comment line.
(d, f, b, i, h, e, a, c, g)
\\\\
\end{enumerate}


 \newpage
%--------------------------------------------------------------------------------------------------

\section*{Problem 7}
\begin{enumerate}[label=(\alph*)]
\item Give the tree resulting from a traversal of the graph below starting at vertex a using BFS and DFS. Assume that the neighbors of a vertex are considered in alphabetical order.\\
\\
\fbox{
 \includegraphics[width=2in]{M6-fig15}\\
}\\\\
{\color{blue}{\bf Figure 15:} \emph{A graph has 7 vertices, a through g, and 10 edges. Vertex e on the left end is horizontally inline with vertex g on the right end. Vertex b is below and to the right of vertex e. Vertex c is above vertex e and to the right of vertex b. Vertex f is between and to the right of vertices c and b. Vertex f is horizontally inline with vertices e and g. Vertex a is above and to the right of vertex f. Vertex d is below and to the right of vertex f. Vertex a is vertically inline with vertex d. Vertex g is between and to the right of vertices a and d. The edges between the vertices are as follows: e and b; b and c; c and f; c and a; a and d; b and f; f and a; f and d; a and g; and d and g.
\\
}
}
\\
\\
%Enter your answer below this comment line.
DFS:\\
(\{a, c\}, \{c, b\}, \{b, e\}, \{b, f\}, \{f, d\}, \{d, g\})\\\\
BFS:\\
(\{a, c\}, \{a, d\}, \{a, f\}, \{a, g\}, \{c, b\}, \{b, e\})
\\\\
\end{enumerate}
 \newpage
%--------------------------------------------------------------------------------------------------

\section*{Problem 8}
An undirected weighted graph G is given below:\\\\
\fbox{
 \includegraphics[width=2in]{M6-fig16}\\
}
\\\\
{\color{blue}{\bf Figure 16:} \emph{An undirected weighted graph has 6 vertices, a through f, and 9 edges. Vertex d is on the left. Vertex f is above and to the right of vertex d. Vertex e is below and to the right of vertex f, but above vertex d. Vertex c is below and to the right of vertex e. Vertex a is above vertex e and to the right of vertex c. Vertex b is below and to the right of vertex a, but above vertex c. The edges between the vertices and their weight are as follows: d and f, 1; d and e, 4; f and e, 2; e and a, 2; f and a, 3; e and c, 5; c and a, 7; c and b, 5; and a and b, 6.
\\
}
}
\\
\\
\begin{enumerate}[label=(\alph*)]
\item Use Prim's algorithm to compute the minimum spanning tree for the weighted graph. Start the algorithm at vertex a. Show the order in which the edges are added to the tree.\\\\
%Enter your answer below this comment line.
Minimum tree in edge order added with weight:\\
1. \{a, e\}: 2\\
2. \{e, f\}: 2\\
3. \{f, d\}: 1\\
4. \{e, c\}: 5\\
5. \{c, b\}: 5\\
Total weight: 15
\\\\
\item What is the minimum weight spanning tree for the weighted graph in the previous question subject to the condition that edge $\{d,\, e\}$ is in the spanning tree?\\\\\\\\
%Enter your answer here.
If \{d, e\} must be added, then we will start there.\\
1. \{d, e\}: 4\\
2. \{d, f\}: 1\\
3. \{e, a\}: 2\\
4. \{e, c\}: 5\\
5. \{c, b\}: 5\\
Total weight: 17
\\\\
\item How would you generalize this idea? Suppose you are given a graph G and a particular edge $\{u,\,v\}$ in the graph. How would you alter Prim's algorithm to find the minimum spanning tree subject to the condition that $\{u,\,v\}$ is in the tree?\\\\
%Enter your answer below this comment line.
I would simply start with the available edges with vertices $u$ and $v$, adding the edge \{u, v\} to the total.
\\\\
\end{enumerate}


\end{document}
